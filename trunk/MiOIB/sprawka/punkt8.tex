\section{Propozycje udoskonaleń i spodziewane efekty}
Przedstawione algorytmy można przede wszystkim bardzo łatwo i skutecznie zrównoleglić na wiele wątków, co zdecydowanie wpłynie na czas obliczeń - więcej rozwiązań można przejrzeć w tym samym czasie, co daje probabilistycznie większe prawdopodobieństwo znalezienia optimum lub wartości zbliżonej do optimum aniżeli wersja jednowątkowa. Oczywiście uzyskany efekt jest zależny w tym wypadku od architektury komputera i nic nie da w przypadku jednoprocesorowego stanowiska obliczeniowego, ale w dzisiejszych czasach nie powinno to stanowić większego problemu.

Można by pokusić się o napisanie prostych funkcji uaktualniających wartość funkcji rozwiązania bez konieczność przeglądania całego rozwiązania, co na pewno umożliwiałoby zyskanie kilku cykli procesora na przeglądnięcie większej liczby rozwiązań. Jednak wydaje się, że uzyskany zysk czasowy nie wpłynąłby znacząco na ostateczne wyniki poszczególnych algorytmów (dotyczyłby algorytmu random, greedy oraz steepest w równym stopniu, więc końcowe wyniki byłyby tylko odpowiednio przeskalowane).

Wybór języka programowania $C++$ był podyktowany m.in. szybkością działania kompilowanego kodu (brak kodu pośredniego, który jest wykonywany przez wirtualną maszynę, tylko kod natywny pod dany system operacyjny), jednak ponownie wydaje się, że ta kwestia ma równorzędny wpływ na uzyskane wyniki każdego z algorytmów.

Na koniec istotne wydaje się odpowiednie dobieranie punktu startowego dla algorytmów lokalnego przeszukiwania zgodnie z zasadą ''dobre rozwiązania powinny znajdować się stosunkowo blisko lepszych rozwiązań'', czyli odpalenie na początek heurystyki znajdującej właśnie dobre rozwiązanie, które będzie stanowiło podstawę do działania dla algorytmów przeszukiwania. Wówczas istnieje szansa, że trafimy na lepsze rozwiązanie z większym prawdopodobieństwem aniżeli startowanie z losowych rozwiązań (być może słabych jeśli chodzi o wartość). Wniosek ten jest prawdziwy, co można zaobserwować w tabeli \ref{tab:heur_local}. Dla porównania, algorytmy steepest i greedy uruchamiane 10-krotnie z losowych punktów startowych uzyskały dla badanej instancji flv55.atsp wyniki odpowiednio 153,7 i 142,6. Uwzględniono w niej wybraną przez nas zachłanną heurystykę, która zgodnie z naszym doświadczeniem okazała się być lepsza niż algorytmy lokalnego przeszukiwania startujące z losowych punktów.

Ale, co właśnie pokazuje przytoczona tabela, nie były to najlepsze wyniki, ponieważ udało się je poprawić przeglądając sąsiedztwa rozwiązań wskazanych przez heurystykę. Stąd płynie dość istotny wniosek, że korzystnie jest, gdy rozwiązanie początkowe, od którego startują algorytmy lokalnego przeszukiwania, jest wybrane w sposób nieprzypadkowy przy pomocy heurystyki, która już posiada dość dobre przybliżenie względem optimum. Oczywiście pojawia się tutaj problem dobrania odpowiedniej heurystyki do problemu i porównania różnych heurystyk między sobą. Jednak można przykładowo rozwiązać go ''automatycznie'' wykorzystując algorytm ewolucyjny (genetyczny), gdzie do populacji początkowej trafiają rozwiązania wygenerowane przez różne heurystyki.

\begin{table}[h!]
	\centering
       \begin{tabular}{|l|l|l|}
        \hline
		Heurystyka & Heurystyka + Steepest & Heurystyka + Greedy \\
		\hline
		139.74 & 126.12 & 126.12 \\
		143.97 & 124.38 & 124.38 \\
		132.52 & 128.61 & 130.16 \\
		125.81 & 118.84 & 118.84 \\
		137.38 & 126 & 127.55 \\
		139.12 & 123.82 & 130.53 \\
		143.97 & 124.38 & 124.38 \\
		139.74 & 126.12 & 126.12 \\
		132.52 & 128.61 & 130.16 \\
		139.18 & 137.69 & 137.69 \\
		\hline
		\end{tabular}
		\caption{Jakość uzyskiwanych rozwiązań przy wykorzystaniu algorytmów przeszukiwania lokalnego startujących z punktu wyznaczonego przez heurystykę dla instancji flv55.atsp.}
		\label{tab:heur_local}
\end{table}