\section{Wnioski}
Zbierzmy obecnie komentarze występujące w ninejszej pracy w całość. Do rozwiązania problemu ATSP najlepszym algorytmem okazała się zachłanna heurystyka. Działa ona najszybciej i zwraca najlepsze wyniki. Algorytmy przeszukiwania lokalnego pod względem jakości nie odbiegają w znaczącym stopniu od heurystyki, jednak dużo dłużej szukają lokalnego optimum. W ogólności jednak heurystyka startująca z danego punktu, działająca w sposób deterministyczny (przeglądając sąsiadów zawsze w tej samej kolejności) może rozwiązania optymalnego nigdy nie znaleźć, natomiast algorytm przeszukiwania lokalnego zawsze może wystartować z punktu leżącego na zboczu, które zaprowadzi go do ekstremum globalnego.

Porównując algorytmy Steepest i Greedy zauważamy, że mimo minimalnej tylko różnicy w kodzie, sprowadzającej się w gruncie rzeczy do dodania jednej instrukcji \texttt{break} w pętli, natura ich działania jest skrajnie różna. Algorytm Greedy dużo szybciej zmienia sąsiadów, natomiast algorytm Steepest przegląda ich dużo większą liczbę.

Nie chcąc powtarzać bardziej szczegółowych wniosków, odsyłamy do poszczególnych punktów ninejszej pracy.