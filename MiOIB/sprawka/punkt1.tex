\section{Rozwiązanie i operatory sąsiedztwa}
\subsection{Reprezentacja rozwiązania}
Dla porządku i nadania sensowności dalszym rozważaniom należy krótko wspomnieć czym jest rozwiązanie problemu i jaką formę reprezentacji tego rozwiązania przyjęliśmy. Rozwiązaniem jak już zostało wspomniane w poprzednim punkcie jest kolejność odwiedzanych miast. Wykorzystujemy przy tym założenie o tym, że dane miasto musi być odwiedzone dokładnie jeden raz, ponieważ mogłoby się zdarzyć, że z miasta A do B jest dalej niż gdyby jechać z miasta A do C a potem do B. Oczywiście w teorii taka sytuacja jest wyeliminowana przez minimalizację funkcji celu (jaką jest sumaryczna długość trasy), która automatycznie wybrałaby wówczas sytuację A-C-B, ale jednak jest to istotne, że korzystamy z oryginalnej macierzy odległości (bo można by ją poddać procesowi poszukiwania najkrótszej ścieżki między każdą parą miast, ale wówczas mogłoby wyjść na to, że dane miasto jest odwiedzone więcej niż jeden raz - a to wszystko dlatego, że nie przyjmujemy żadnych założeń na temat odległości między miastami)

Stąd rozwiązanie to po prostu tablica jednowymiarowa o długości $n$ zawierająca na kolejnych pozycjach miasta, które powinny być odwiedzone w kolejności wynikającej z tej tablicy. Funkcję celu można szybko wyliczyć sumując odległości między kolejnymi parami miast występującymi w tabeli z uwzględnieniem połączenia między ostatnim i pierwszym miastem.
\subsection{Operator 2-OPT}
Wykorzystaliśmy najprostszy z operatorów sąsiedztwa, jakim jest 2-OPT. Jego zasada generowania sąsiadów jest bardzo prosta - wybieramy dwa losowe punkty w rozwiązaniu i zamieniamy ze sobą miejscami. W ten sposób uzyskujemy nowe rozwiązanie (być może lepsze, być może gorsze). Bardzo łatwo można wykazać, że dla danego rozwiązania można wygenerować $\frac{n(n-1)}{2}$ sąsiadów. Wynika to z tego, że wpierw wybieramy 1 z $n$ pozycji rozwiązania, następnie 1 z $n-1$ pozostałych miejsc i je zamieniamy. Ponieważ przykładowo zamiana miejsca 3 z miejscem 8 da takiego samego sąsiada jak zamiana miejsca 8 z miejscem 3, to dzielimy uzyskaną liczbę przez 2.